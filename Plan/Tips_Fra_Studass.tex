




Viktig at vi forstår det som står i oppgave 2. - Forstå den teorien der!


Formål for eksperimentet
    Modell av virkeligheten basert på teori
        Beskrive virkeligheten ved den modellen

Overblikk over systemet og formler
    Ser på fysiske størrelser
        Se om hvilke er kjente, hvilke må måles. 

Poenger med å skrive en ordentlig plan: 
    Unngå å gjøre mye om igjen. 
    Og endre på måten vi vil gjøre en måling, dersom vi ser noe blir bedre. 

Praktisk utførelse: 
    Punktliste over alt vi skal gjøre
    Hvilke størrelser som skal måles
        Hvordan vi skal måle de som måles. 
    At vi har en måte å måle alle størrelsene vi ønsker å måle. 
    Det er bra å ha flere måter å måle størrelsene, fordi kontrollmåling er bra! 
    Modell/figur laget i vektorgrafikk. Inkscape
    Må skries i Latek med den malen som ligger ute
    

Usikkerhetsanalyse
    Fysiske størrelser i systemet vårt. 
    Hvordan vi måler disse vil det påvirke usikkerheten til disse størrelsene.
        Hvordan måler vi denne usikkerheten? 
    Hvilke usikkerhet vil vi få for hver størrelse?

    Hvilke farer, hvilke feilkilder. 

Numerikk. 
    Bruke newtons andre lov til å finne differensialligningen. Skrive den som et sett av første-ordens differensialligning. 
    Hvordan vi løser den? -> Eulers metode
    Hvordan det numeriske kommer til å bli regnet ut? 
        Hvordan fungerer eulers metode? 


    
    

    
Tilbakemelding til studenten
14.09.17 00:52
1.

Det hadde nok vært lurt med en referanse i dette tilfellet siden det ikke er helt riktig som dere skriver. Dette gjelder kun hvis det stive legemet som ruller er en ring. Generelt er rotasjonsenergien gitt av

1/2*I*omega^2.

I tillegg er det jo slik at energien E er den potensielle energien + den kinetiske. For å se på endringen i energi kan dere jo da også ha en kinetisk og potensiell energi både i slutt-tilstanden og i start-tilstanden.

Det gir ikke mening å samle friksjonskreftene i et symbol da friksjonskraften og luftmotstanden oppfører seg på forskjellige måter. Dere har også ikke tatt med normal-kraften. Dere bør presentere de ferdig dekomponerte likningene for denne delen og også Newtons 2. lov for rotasjon blir viktig i og med kravet for rulling uten slipping som blir essensielt (se oppg. 2 tekst). 

2.

Denne delen er alt for kort. Den skal være et detaljert punktliste over hvordan dere skal gå frem for å måle alle de relevante fysiske størrelsene. Det kan også være lurt å finne flere forskjellige måter å måle disse størrelsene på.

3.

Her må dere være spesifikke. Når dere har skrevet hvordan dere skal måle høyden i del 2. kan dere gjøre et overslag på hvor nøyaktige dere vil være.

Det samme gjelder for de andre avsnittene. Dere må være spesifikke på størrelsene som dere skal måle og oppgi grundig hvordan usikkerheten skal bli målt. Hvis dere har en utregnet størrelse, ja så skal dere regne ut gauss feilforplantningslov på planen så den er klar. Skal dere bruke standard feil, så spesifiser hvilke størrelser dette gjelder for.

4. 

Her mangler dere å oppgi spesifikt den annen-grads formelen dere skal løse numerisk. Hva er [PH]?