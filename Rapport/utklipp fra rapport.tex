Friksjonskraften $F_{F}$ virker på ballen med retning motsatt av rulleretningen. Normalkraften $F_{N}$ virker på ballen i y-retning. Newtons første lov sier at summen av kreftene i y-retning er null, og at det kan løses for $F_{N}$ slik at
\begin{equation}
F_{N} = mg\cdot\cos(\alpha)
\end{equation}
Friksjonskraften er definert som 
\begin{equation}
F_{F} = \mu\cdot F_{N}
\end{equation}
Hvor $\mu$ er friksjonskoeffisienten mellom ballen og skråplanet. 

$\kappa = \frac{1}{R}$, der $\kappa$ er krumningen til sirkelbanen og R er radiusen til sirkelbanen.



\autoref{tab:TrackerTabell}
\FloatBarrier
 \begin{table}[h!]
 \begin{center}
 \begin{tabular}{c c c c}
 \hline
 Forsøk & $x_{Start}$ (cm) & $x_{Slutt}$ (cm) & $\mu_E (10^{-3})$ \\
 \hline

 1 & -53,546 & -45,290 & 1,4234\\ 
 2 & -53,021 & -44,982 & 1,3860\\
 3 & -53,013 & -44,685 & 1,4358\\
 4 & -52,819 & -44,474 & 1,4387\\
 5 & -53,286 & -45,426 & 1,3551\\
 6 & -52,902 & -44,857 & 1,3870\\
 7 & -52,643 & -44,546 & 1,3960\\
 8 & -52,894 & -44,593 & 1,4312\\
 9 & -53,196 & -45,056 & 1,4034\\
 10 & -52,806 & -44,785 & 1,3829\\[1ex]
 \hline
 
\end{tabular}
\end{center}
\caption{Posisjon i x-retning ifra Tracker.}
\label{tab:TrackerTabell}
\end{table}
\FloatBarrier


$$\mu_E = {\overline{\mu}}\pm m$$
Da blir $\overline{\mu_E}$
$$\overline{\mu_E} = 1,40 * 10^{-3}$$
Og


%Usikkerheten i de målte størrelsene kan ikke allene stå for avviket til antatt verdi. Avviket kan heller ikke begrunnes med at vi gjorde for mange antagelser. Det er fordi alle antakelsene vil medføre høyere rullefriskjonskoeffisient enn faktisk verdi. Dermed er det ukjente årsaker.

%https://no.sharelatex.com/project/59afd049ca71f758b7f7e1fe

%Et oppfølgende eksperiment kan utføres for å beregne luftmotstand til en rullende kule og gi en mer nøyaktig verdi for rullefriksjonen. \\


%Det er en usikkerhet i målingen av friksjonskoeffisienten. kommer fra usikkerheter i målinger av høyden til- og bredden mellom festepunktene til sirkelbanen. Da radien til sirkelbanen er beregnet fra disse verdiene får vi en usikkerhet i sirkelbuen. 
% $\mu$ er i dette eksperimentet avhenging av størrelsene $\Delta x$ og $R$, som vist i ligning \eqref{eq:Utrykk for my}.
%%%%%%%
%Siden $\delta R >> \delta x$ påvirker ikke $\delta x$ usikkerheten til $\mu$ i noe særlig grad og nærmest all usikkerheten kommer derfor fra usikkerheten i radiusen til banen. Skulle eksperimentet blitt utført mer nøyaktig burde startoppsettet bestått av en sirkelskive med gitt radius, og ikke bare et utsnitt av en sirkelbane, som vist i figur \autoref{fig:pendel}. Dette ville ha redusere usikkerheten til $\mu$ kraftig, da radien ville hatt et mye mindre avvik. Til tross for dette er usikkerheten til $\mu_{E} \pm 4\%$.
%%%%%
%Usikkerheten til radiusen av sirkelbanen er gitt av hvor nøyaktig vi klarte å sette endepunktene til sirkelbanen i eksperimentet. Vi bergegner usikkerheten i høyden i ytterpunktene til ±1.mm. Siden punktene ligger $60\pm 0.5$ cm for bunnpunktet og høyden var satt til 5cm, får vi at radien til banen har en usikkerhet på  $\pm 9 cm$.
%%%%%



%Det er også viktig å vite hvor mange signifikante sifre beregningen vår av $\mu$ har. 
%De to målte størrelsene $\mu_E$ er avhengig av har kun to signifikante sifre siden usikkerheten ligger i det andre sifferet. $\Delta x \approx 8,1cm \pm 0,2 cm$ og $R \approx 363cm \pm 9cm$.
%%For å unngå avrundingsfeil ble 5 signifikante uavrundede siffer brukt for de to avhengie størrelsene helt frem til siste avrunding av $\mu_E$.\\


%Differansen mellom $\mu_N$ og $\mu_E$ er $2,55 \cdot 10^-3$ eller 15,4\% av $\mu_N$. Denne store differansen kan skyldes antagelsene gjort for å beregne $\mu_E$. Spriket kan også skyldes at den numeriske modellen beregner strekningen kulen faktisk beveger seg. kulen følger i virkeligheten en parabel mens den numerisk er sammenliknet med kun bevegelse langs x-aksen. Denne er feilen er dog ikke så stor da vinkelen til enhver tid er $\lesssimm 10\degree$. Det betyr at dersom kulen befinner seg ved $x = 50 \hspace{1mm}\mathrm{cm}$ vil den dersom banen var rettlinjet vil strekningen langs planet vært \\
%$s = \sqrt{50^2 + (50 \cdot tan(10))^2)} \hspace{1mm}\mathrm{cm} = 50,77 \hspace{1mm}\mathrm{cm}$ , Altså en feil på $ \sim 1,5 \%$.\\%

%Ved Newtons andre lov blir kreftene som virker på kulen beskrevet av en differensialligning. Kulen beveger seg i en sirkelbane med liten krumning. Dermed kan differensialligningen tilnærmes til en lineær differensialligning, som lar seg løse analytisk, uten store feil. Tapet i utslaget fra likevektsposisjoen blir da proporsjonalt med friksjonskoeffisienten.